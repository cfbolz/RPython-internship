\documentclass{beamer}

% Python listing setup

\usepackage{color}
\usepackage[procnames]{listings}
\usepackage{textcomp}
\usepackage{setspace}
\usepackage{palatino}
\renewcommand{\lstlistlistingname}{Code Listings}
\renewcommand{\lstlistingname}{Code Listing}
\definecolor{gray}{gray}{0.5}
\definecolor{green}{rgb}{0,0.5,0}
\definecolor{lightgreen}{rgb}{0,0.7,0}
\definecolor{purple}{rgb}{0.5,0,0.5}
\definecolor{darkred}{rgb}{0.5,0,0}
\lstnewenvironment{python}[1][]{
\lstset{
language=python,
basicstyle=\ttfamily\small\setstretch{1},
stringstyle=\color{green},
showstringspaces=false,
alsoletter={1234567890},
otherkeywords={\ , \}, \{},
keywordstyle=\color{blue},
emph={access,and,as,break,class,continue,def,del,elif,else,%
except,exec,finally,for,from,global,if,import,in,is,%
lambda,not,or,pass,print,raise,return,try,while,assert},
emphstyle=\color{orange}\bfseries,
emph={[2]self},
emphstyle=[2]\color{gray},
emph={[4]ArithmeticError,AssertionError,AttributeError,BaseException,%
DeprecationWarning,EOFError,Ellipsis,EnvironmentError,Exception,%
False,FloatingPointError,FutureWarning,GeneratorExit,IOError,%
ImportError,ImportWarning,IndentationError,IndexError,KeyError,%
KeyboardInterrupt,LookupError,MemoryError,NameError,None,%
NotImplemented,NotImplementedError,OSError,OverflowError,%
PendingDeprecationWarning,ReferenceError,RuntimeError,RuntimeWarning,%
StandardError,StopIteration,SyntaxError,SyntaxWarning,SystemError,%
SystemExit,TabError,True,TypeError,UnboundLocalError,UnicodeDecodeError,%
UnicodeEncodeError,UnicodeError,UnicodeTranslateError,UnicodeWarning,%
UserWarning,ValueError,Warning,ZeroDivisionError,abs,all,any,apply,%
basestring,bool,buffer,callable,chr,classmethod,cmp,coerce,compile,%
complex,copyright,credits,delattr,dict,dir,divmod,enumerate,eval,%
execfile,exit,file,filter,float,frozenset,getattr,globals,hasattr,%
hash,help,hex,id,input,int,intern,isinstance,issubclass,iter,len,%
license,list,locals,long,map,max,min,object,oct,open,ord,pow,property,%
quit,range,raw_input,reduce,reload,repr,reversed,round,set,setattr,%
slice,sorted,staticmethod,str,sum,super,tuple,type,unichr,unicode,%
vars,xrange,zip},
emphstyle=[4]\color{purple}\bfseries,
upquote=true,
morecomment=[s][\color{lightgreen}]{"""}{"""},
commentstyle=\color{red}\slshape,
literate={>>>}{\textbf{\textcolor{darkred}{>{>}>}}}3%
         {...}{{\textcolor{gray}{...}}}3,
procnamekeys={def,class},
procnamestyle=\color{blue}\textbf,
framexleftmargin=1mm, framextopmargin=1mm, frame=shadowbox,
rulesepcolor=\color{blue},#1
}}{}



\usepackage{beamerthemesplit} % Activate for custom appearance
\usetheme{Warsaw}



\newtheorem{question}{Question}

\title{RPython}
\author{Leonard de HARO}
\date{\today}

\begin{document}

\frame{\titlepage}

\section[Outline]{}
\frame{\tableofcontents}

\section{Introduction}
\frame
{
  \frametitle{The Pypy Project}

  \begin{itemize}
 	 \item<1-> New Python VM, written in RPython\\
  			\uncover<3->{\emph{The Interpreter}}
  	\item<2-> New language to design VMs: RPython\\
  			\uncover<3->{\emph{The Translation Toolchain}}
  \end{itemize}
  
\uncover<4->
{   
  \begin{fact}
  The Translation Toolchain provides a JIT compiler on demand!
  \end{fact}
 } 
}

\section{Just-In-Time compilation}

\frame
{
  \frametitle{Sure... But what's a JIT compiler?}
  
  Two kinds:
	\begin{itemize}
		\item Method JITs \\
			\textit{e.g.} HotSpot in JVM)
		\item Tracing JITs \\
			\textit{e.g.} Pypy
	\end{itemize}
}

\frame
{
  \frametitle{Method JITs}

	\begin{itemize}
	  	\item<1-> Works on the bytecode (linear)
	  	\item<2-> Notices "Hot Spots"
	  	\item<3-> Compiles them (native code)
	  	\item<4-> Uses the native code version
	\end{itemize}
  
}

\frame
{
  \frametitle{Tracing JIT}
  	
	\begin{itemize}
		\item<1-> Works during execution
		\item<2-> Finds a "hot loop"
		\item<3-> Traces the execution
		\item<4-> Optimizes it (including guards)
		\item<5-> Uses the optimized traced version
	\end{itemize}
\uncover<6->
{
	\begin{fact}
	Pypy is a \textbf{meta-tracing JIT} compiler: feed it a properly annotated interpreter, it gives you back a tracing JIT	 interpreter.
	\end{fact}
}
}

\frame
{
	\begin{question}
	Examples of Pypy's JIT all work with bytecode. Can we make it work on ASTs?
	\end{question}
}		


\section{RPython as a language}
\frame
{
  \frametitle{General properties of RPython}
  
  \begin{itemize}
  	\item<1-> Strict and valid subset of Python
	\item<2-> Statically typed (with exception)
	\item<3-> Output C (when used in TT)
	\item<4-> \textbf{Creates JITing VMs} (for under 10 lines of code)
	\item<5-> Still in development although useable
  \end{itemize}
}

\frame
{
  \frametitle{Writing a JIT VM}
  
  \begin{itemize}
  	\item<1-> Write your interpreter
	\item<2-> Add RPython instructions for translation
	\item<3-> Add RPython instructions for JITing
		\begin{itemize}
			\item<4-> \texttt{can\_enter\_jit}
			\item<5-> \texttt{jit\_merge\_point}
			\item<6-> Declare \textit{Red} and \textit{Green} variables
		\end{itemize}
	\item<7-> Optimize (\textit{e.g.} insert \textit{assert} to help the interpreter or  use fixed-size lists)
  \end{itemize}
}

\section{Demonstration}
\frame
{
\begin{center}
\textbf{Demo Time !}\\
(see Andrew Brown's tutorial on Pypy's Blog)
\end{center}
}


%\begin{frame}[fragile]{Some Python Code}
%\begin{python}
%def foo(x):
%	if x==0:
%		return 42
%	else:
%		return "42"
%\end{python}
%\end{frame}

\end{document}
